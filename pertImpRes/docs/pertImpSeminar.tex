\documentclass{beamer}
\usepackage{beamerthemeshadow}
\usepackage[utf8]{inputenc}
\usepackage{listings,xcolor}

\usepackage{moreverb}
\usepackage{ulem}
\usepackage{soul}
%\usepackage{algorithmicx}
%\usepackage{algpseudocode}
\usepackage{pseudocode}
% \usepackage{hyperref}
% \usepackage{color}
% \definecolor{light}{gray}{.80}
% \usepackage{graphicx}
% \usepackage{amsfonts}
% \usepackage{amsmath}
% \usepackage{amssymb}
% \usepackage{amsthm}
% \usepackage{algorithmicx}
% %\usepackage{program}
% \usepackage{rotating}
\usepackage{natbib}
%\usepackage{authordate1-4}

\newcommand{\va}[1]{\log(x_{1#1})}
\newcommand{\vb}[1]{\log(y_{1#1})}
\newcommand{\vc}[1]{\log(x_{2#1})}

%\newcommand{\tlam}{\boldsymbol{\lambda}}
%\newcommand{\tnu}{\boldsymbol{\nu}}
\newcommand{\bz}{\boldsymbol{z}}
\newcommand{\tmu}{\boldsymbol{\mu}}
%\newcommand{\yell}{\mathbf{l}}
%\newcommand{\kay}{\mathbf{k}}
\newcommand{\compArg}{x_1,\ldots,x_d}
\newcommand{\theComp}{
f(y_1,\ldots,y_m),\,\, \text{and}\,\,  g^1(\compArg),\ldots,g^m(\compArg)\\
h(\compArg) = f(g^1(\compArg),\ldots,g^m(\compArg))
}
\newcommand{\prt}[1]{\left(\frac{\partial }{\partial x_#1}\right )^{\psi_#1}}

\newcommand{\xt}{\mathcal{E}_t}
\newcommand{\omTerms}[1]{
  \faDiPyr{2}{\bCompAug{#1}}   \gMultTerm{2}{\bCompAug{#1}}+ \ldots +
  \faDiPyr{k-1}{\bCompAug{#1}} \gMultTerm{k-1}{\bCompAug{#1}}}
\newcommand{\mma}{{\bf Mathematica}}
\newcommand{\mlb}{{\bf MATLAB}}
\newcommand{\bComp}[2]{\overset{\circlearrowleft(#1)}{\mathcal{B}_{#2}}}
\newcommand{\bCompAug}[1]{\overset{\circlearrowleft(#1)}{\mathcal{B}_{Aug}}}
\newcommand{\mkMat}[2]{\overset{\circlearrowleft(#2)}{
\underset{(#1)}{\mathcal{M}}}}
\newcommand{\nkMat}[2]{\overset{\circlearrowleft(#2)}{
\underset{(#1)}{\mathcal{N}}}}
\newcommand{\bmatObj}{\item[bMat] $B \in \rSpTwo{L}{L\cdot \tau}$}
\newcommand{\bCompMultsObj}{\item[gMults] $M_k \in \rSpTwo{L \theta \dtg}{L \dtg}$}
\newcommand{\gMultsObj}{\item[gMults] $W \in \rSpTwo{\idtg}{\dtg}$}
\newcommand{\lilGMultsObj}{\item[lilGMults] $W \in \rSpTwo{\dtg}{\dtg}$}
\newcommand{\bPyrsObj}{\item[bPyrs] $B_\Delta \in \rSpTwo{\fv}{s}$}
\newcommand{\gPyrsObj}{\item[gPyrs] $G_\Delta \in \rSpTwo{\fv}{s}$}
\newcommand{\vbar}{{\mid}}
\newcommand{\rSpTwo}[2]{\mathcal{R}^{#1}\times\mathcal{R}^{#2}}
\newcommand{\sdefLine}[4]{
{ #1}{\small ($#3$)}&  #2&#4\\}
\newcommand{\pscl}[2]{\frac{(#1+#2)!}{(#1!#2!)}}
\newcommand{\fv}{\mathcal{N}}
\newcommand{\theLeads}{\fv_+}

\newcommand{\leadsNeededDef}{ List of ordered pairs. Ordered pair consists of variable number followd by the lead present. Each lead variable appearing in the model
has an ordered pair. $\{ (v_1,l_1),\ldots,(v_{\fv_+},l_{\fv_+}) \}$
}
\newcommand{\neqDef}{$L$ -- the number of equations}
\newcommand{\theInOuts}[2]{
\centerline{{\bf#1}}
\begin{description}
#2
\end{description}
}

\newcommand{\theIns}[2]{\theInOuts{#1 --- The Inputs}{#2}}
\newcommand{\theOuts}[2]{\theInOuts{#1 --- The Outputs}{#2}}



\newcommand{\algDef}[1]{
\subsection{#1}
\label{sec:#1}
}
\newcommand{\mstr}[2]{
{ \scriptstyle #1}\overset{#2}{
\begin{bmatrix}
  \ 
\end{bmatrix}
}}
\newcommand{\bk}[2]{B_{#1}^{#2}}
\newcommand{\pyr}[2]{#1_{#2}}
\newcommand{\altF}{\mathcal{F}}
\newcommand{\dtg}{\mathcal{V}}
\newcommand{\idtg}{\mathcal{V}_I}
\newcommand{\gMult}{\mathcal{W}}
\newcommand{\gMultTerm}[2]{\underset{(#1)}{\mathcal{W}} \left (#2 \right )}
\newcommand{\tlam}{\boldsymbol{\lambda}}
\newcommand{\tnu}{\boldsymbol{\nu}}
\newcommand{\yell}{\mathbf{l}}
\newcommand{\kay}{\mathbf{k}}



%from practicalConsiderationsSeminar.tex
\newcommand{\inv}[1]{#1^{-1}}

\newcommand{\stitle}[1]{{\centerline{
      \begin{minipage}{4.0in}
        \begin{center}
\large\bf #1
        \end{center}
      \end{minipage}
}}} 

\newcommand{\sdef}[3]{
%\nwslide
\stitle{#1$(#2)$}
\begin{gather*}
#2=
#3
\end{gather*}
}

\newcommand{\nlsys}{\mathcal{H}}
\newcommand{\eqsys}{\mathcal{G}}
\newcommand{\fprod}{\mathcal{Y}}
\newcommand{\fga}[1]{\underset{(#1)}{\mathcal{F}}}
\newcommand{\pyrStar}[1]{#1^\ast_{{\bf pyr}}}
\newcommand{\pyrStarF}[1]{#1^\ast_{{\bf pyr_1}}}
\newcommand{\pyrStarS}[1]{#1^\ast_{{\bf pyr_2}}}
\newcommand{\eqcon}{\mathsf{m}}
\newcommand{\eqsoln}{{h}}
\newcommand{\seqsys}{\mathcal{G}_s}
\newcommand{\seqcon}{\mathsf{m}_s}
\newcommand{\seqsoln}{{h}_s}
\newcommand{\olrm}{\mathcal{W}_l}
\newcommand{\sfga}{\mathcal{F}_s}
\newcommand{\solrm}{\mathcal{W}_{ls}}


%from practicalConsiderationsSeminar.tex




\newcommand{\bce}{B_{CE}}
\newcommand{\bstoch}{B_{Stoch}}
\newcommand{\bstsub}[1]{B_{Stoch,#1}}
\newcommand{\pds}[1]{\frac{\partial #1}{\partial \sigma}}
\newcommand{\pdds}[1]{\frac{\partial^2 #1}{\partial \sigma^2}}



\newcommand{\bigB}{\mathcal{B}}
\newcommand{\pdOne}[2]{\frac{\partial #1}{\partial #2}}
\newcommand{\pdTwo}[2]{\frac{\partial^2 #1}{\partial #2^2}}
\newcommand{\pdOneOne}[3]{\frac{\partial^2 #1}{\partial #2\partial #3}}
\newcommand{\doPyrOne}[2]{\Delta_{#1}(#2)}
\newcommand{\doPyrTwo}[2]{\Delta^2_{#1}(#2)}
\newcommand{\theVars}{\{\bar{x},\epsilon,\sigma\}}

%\newcommand{\expt}[2]{\mathcal{E}_{#1} \left ( {#2} |\mathcal{I}_{#1}\right )}
\newcommand{\expt}[2]{{E}_{#1} \left  [ {#2}  \right  ]}
\newcommand{\theB}{\mathcal{B}}
\newcommand{\justX}{x_{t-\tau},\ldots,x_{t-1}}
%\newcommand{\theArg}[1]{x_{t-\tau},\ldots,x_{t-1},#1,\sigma}
\newcommand{\theArg}[1]{\justX,#1,\sigma}
\newcommand{\theArgVert}{x_{t-\tau} \\ \vdots \\ x_{t-1}}
\newcommand{\real}[1]{\mathcal{R}^{#1}}
\newcommand{\faDiPyr}[2]{\underset{(#1)}{\Delta} \left (#2 \right )}

\newcommand{\mi}[3]{#1[\{#2\}(#3)}
\newcommand{\pDrv}[2]{\left ( \frac{\partial}{\partial #1} \right )^{#2}}
\newcommand{\spDrv}[2]{\left ( \frac{\partial#1}{\partial #2} \right )}
\newcommand{\vSum}[1]{\sum_{i_#1=0}^\infty}
\newcommand{\vEps}[1]{\epsilon_{t-i_#1}}
\newcommand{\faa}{Fa\`{a} di Bruno}
\newcommand{\myamp}{&}
\newcommand{\mywedge}{^}
\newcommand{\cFunc}{{\bf c}}
\newcommand{\kFunc}{{\bf k}}
\newcommand{\zFunc}{{\bf z}}
\newcommand{\fwt}[2]{\frame{\frametitle{#1} #2}}
\theoremstyle{definition}
\newtheorem{defn}{Definition}[section]
\newcommand{\sNum}[2]{(#1 \cdot 10^{#2})}


\begin{document}
\title[Taylor Series-Based Ergodic Distribution Moments]{A \st{Perturbation}Taylor Series-{Based} Method for Approximating Dynamic Nonlinear Model Impulse Response Functions and Ergodic Distribution Moments }
\author{Gary S. Anderson}
\date{\today} 

\frame{\titlepage} 

%\frame{\frametitle{Table of contents}\tableofcontents} 
%\input{
%\input{solnOutline.tex}
%\input{polyApprox.tex}
%\input{comparison.tex}	
%\input{conclu.tex}
%\input{doNeoPert.tex}
\begin{itemize}
\item Coordinate symbolic/numeric computing
\item use $E(x_{t+k})$ identity
\item $\gamma \ne 1$
\item details of symbolic expressions for integrals
\item perturbation and projection get same moments
\item LA ticket for August too
\item edo model
\item timing
\item comparisons with lan meyer gohde
\item do burnside reconcile values
\item symbolically compute perturbation or projection solution for neoclassical growth model  compute moments symbolically
\item reproduce burnside like result for neoclassical model
\end{itemize}


\section{Time Invariant Polynomial Functions} 


\frame{\frametitle{Time Invariant Polynomial Functions}


  \begin{itemize}
  \item Consider functions of the form
\begin{gather}
  x_t= G(x_{t-\tau},\ldots,x_{t-1},\epsilon_t) \label{eqnSys}
\end{gather}
 where $G$ is a kth order polynomial in $x_{t-\tau},\ldots,x_{t-1},\epsilon_t$
\item Assume 
  \begin{itemize}
  \item $x_t$ finite dimensional
  \item $G$ time invariant
  \item The system \ref{eqnSys} is stochastically stable\cite{tong90}
  \end{itemize}


  \end{itemize}


}

\begin{frame}
  \frametitle{Example: Burnside Financial Model}
  

  \begin{itemize}
  \item \cite{burnside}
    \begin{gather*}
y_t=\beta(\exp^{(\theta(x_{t+1}))} (1+y_{t+1}))\\
(x_t-\bar{x}) = \rho (x_{t-1}-\bar{x})   + \epsilon_t 
    \end{gather*}
  \end{itemize}


\end{frame}

\fwt{Example: Neoclassical Growth Model's Perturbation Solution}{
The first order conditions lead to 
\begin{gather}
E_t \label{neoEuler}
\begin{bmatrix}
1-\left (  \frac{c_{t+1}^{-\gamma}}{c_t^{-\gamma}}(\alpha e^{z_{t+1}} k_{t}^{\alpha-1} + (1 - \delta)) \right )\\
 c_t +k_{t} - (e^{z_t}k_{t-1}^\alpha + (1-\delta)k_{t-1})\\
z_t - ( \rho_zz_{t-1} +\epsilon_t)
\end{bmatrix}
\intertext{ With Parameter Values}
\alpha = .36, \beta = .99, \delta =1, \gamma = 1,    \rho_z = .95\intertext{ the non-stochastic steady state is}
c^\ast = 0.360247, k^\ast=0.202639, z^\ast=0.
\end{gather}
}
% \frame{\frametitle{ With Parameter Values}
% \begin{gather*}
% \alpha = .36, \beta = .99, \delta =1, \gamma = 1,    \rho_z = .95\intertext{ the non-stochastic steady state is}
% c^\ast = 0.360247, k^\ast=0.202639, z^\ast=0.
% \end{gather*}





% }

\fwt{Obtaining a Perturbation Solution}{{\small
  \begin{itemize}
  \item make the following substitutions  into \ref{neoEuler}
\begin{gather*}
E_t c_{t+1}(\hat{k}_{t-1},z_{t-1},\epsilon_{t},\sigma)= 
\cFunc(\kFunc(\hat{k}_{t-1},z_{t-1},\epsilon_{t}),
\zFunc(\hat{k}_{t-1},z_{t-1},\epsilon_{t}),\sigma \epsilon_{t+1})\\
E_t k_{t+1}(\hat{k}_{t-1},z_{t-1},\epsilon_{t},\sigma)= 
\kFunc(\kFunc(\hat{k}_{t-1},z_{t-1},\epsilon_{t}),
\zFunc(\hat{k}_{t-1},z_{t-1},\epsilon_{t}),\sigma \epsilon_{t+1})\\
E_t z_{t+1}(\hat{k}_{t-1},z_{t-1},\epsilon_{t},\sigma)= 
\kFunc(\kFunc(\hat{k}_{t-1},z_{t-1},\epsilon_{t}),
\zFunc(\hat{k}_{t-1},z_{t-1},\epsilon_{t}),\sigma \epsilon_{t+1})
\end{gather*}

\item  differentiates twice and solve the resulting nonlinear system. 
\item It {\bf seems natural to use this same implicit second order polynomial to compute the time $t+1$ value} in an impulse response  calculation exercise.
\item The ImpulseResponseErgodic Mathematica Packages implements this idea
  \end{itemize}
}
}

\frame{\frametitle{Features of the ImpulseResponseErgodic Package}







  \begin{itemize}
  \item Symbolic calculations compute moments without integration
  \item Can generate a polynomial representation
  \item Can compute Volterra series representation
  \item Arbitrary order
  \item Arbitrary linearization point
  \end{itemize}

{\small Code will soon be available on gitHub es335mathwiz }

}





\frame{\frametitle{Model Definition}
{\small
  \listinginput{1}{neoKeynesianEqnsSubs.mth}
}
}

\frame{\frametitle{Perturbation at Steady State}
{\small
\listinginput{1}{neoPertSolnAtSS.mth}
}
}




\fwt{Time Invariant Polynomial}{
One  easily obtains the second order 
perturbation solution centered at the non-stochastic steady
state.%\footnote{For $\gamma=1=\delta$ there is no variance correction for time t.}
\begin{gather*}
  \begin{bmatrix}
    c_t\\k_t\\z_t
  \end{bmatrix}=
\begin{bmatrix}
\cFunc(\hat{k}_{t-1},z_{t-1},\epsilon_{t})\\ 
\kFunc(\hat{k}_{t-1},z_{t-1},\epsilon_{t})\\
\zFunc(\hat{k}_{t-1},z_{t-1},\epsilon_{t})  
\end{bmatrix}= 
\\
\begin{bmatrix}
 0.3602 + 0.64\hat{k}_{t-1} - 
  1.0107\hat{k}_{t-1}^2 +  0.3422zz_{t-1} + \\0.608\hat{k}_{t-1}zz_{t-1} + 
  0.16256zz_{t-1}^2 + \\0.3602\epsilon_{t} + 
  0.64\hat{k}_{t-1}\epsilon_{t} + 0.34223zz_{t-1}\epsilon_{t} + 
  0.18012\epsilon_{t}^2 \\
0.2026 + 
  0.36\hat{k}_{t-1} - 0.568\hat{k}_{t-1}^2 - 
   + 0.1925zz_{t-1} + 
  0.342\hat{k}_{t-1}zz_{t-1} +\\ 0.0914zz_{t-1}^2 + 
  0.2026\epsilon_{t} + 0.36\hat{k}_{t-1}\\
   \epsilon_{t} + 0.1925zz_{t-1}\epsilon_{t} + 
  0.1013\epsilon_{t}^2\\
  0.95zz_{t-1} + \epsilon_{t}
\end{bmatrix}
\end{gather*}
}








\section{Polynomial Time Series Functional Composition/Iteration}



\frame{\frametitle{Polynomial Time Series Functional Composition/Iteration}

  \begin{itemize}
  \item Compositions of the G functions give leads of model variables
  \item For scalar $x_t$:
\begin{gather*}
x_t=  g(x_{t-1},\epsilon_t)\\
x_{t+1}=  h_1(x_{t-1},\epsilon_t,\epsilon_{t+1})= g(g(x_{t-1},\epsilon_t), \epsilon_{t+1})\\ 
x_{t+2}=  h_2(x_{t-1},\epsilon_t,\epsilon_{t+1},\epsilon_{t+2})= g(g(g(x_{t-1},\epsilon_t), \epsilon_{t+1}) \epsilon_{t+2})\\
\vdots
\end{gather*}
\item   The derivatives of $g$ to nth order determine the derivatives of $h_k$ to nth order
\item Perturbation solution technique exploits this relations to substitute out lead variables to get an approximation $\hat{g}$
  \end{itemize}
}
\frame{\frametitle{Problems Using the Approximation}

  \begin{itemize}
  \item To compute impulse responses, researchers typically iterate using $\hat{g}$
\begin{gather*}
x_{t+2}=  \hat{h}_2(x_{t-1},\epsilon_t,\epsilon_{t+1},\epsilon_{t+1})= \hat{g}(\hat{g}(\hat{g}(x_{t-1},\epsilon_t), \epsilon_{t+1}) \epsilon_{t+2})\\
\vdots \\
x_{t+k}=  \hat{h}_k(x_{t-1},\epsilon_t,\epsilon_{t+1},\cdots,\epsilon_{t+k})= \hat{g}(\hat{g}(\hat{g}(x_{t-1},\epsilon_t), \epsilon_{t+1}) \epsilon_{t+2})\\
\vdots
\end{gather*}
  \item Given this approach, the quality of the approximation for $h_k$ decreases dramatically with k
  \item ``spurious'' steady states require pruning
  \item alternatively one could consider approximating $\hat{h}_k$ at the same time as $\hat{g}$
  \end{itemize}
}
 \fwt{Example With a Long Lead}
 {
  \begin{itemize}
   \item Augment equation \ref{neoEuler} with $\mathbf{c}tp20_t = c_{t+20}, \mathbf{k}tp20_t = k_{t+20}$
    \end{itemize}
{\tiny
 \begin{gather*} 
E_t(     \mathbf{c}tp20_t)= 0.3602 + (9 \times  10^{-9}) k_{t-1} -1.9 \times 10^{-9} k_{t-1}^2 + 3.74533 \sigma^2 + .2 z_{t-1} +\sNum{8}{-9} k_{t-1} z_{t-1} +\\
 .2 \epsilon_t  +\sNum{8}{-10} k_{t-1} z_t + \sNum{1.3}{-1} z_{t-1} \epsilon_t\\
E_t(       \mathbf{k}tp20_t) = 0.2026 + (5 \times  10^{-9}) k_{t-1} -1.1 \times 10^{-9} k_{t-1}^2 + .108 \sigma^2 + .11 z_{t-1} +\sNum{4.5}{-9} k_{t-1} z_{t-1} +\\
 .1 \epsilon_t  +\sNum{4.46}{-10} k_{t-1} z_t 
 \end{gather*}
}
    \begin{itemize}
  \item can provide a second order approximation for the
mean in the distant future
 \item Note how the initial conditions have a diminished 
 impact on the expected future
 values for the model variables.
\item This could be a candidate for what
some refer to as the stochastic steady state.


% % Once using the stochastic steady state, it's possible to approximate variances and other moments.
% % It's straightforward to compute the mean variance skewness 
% % and kurtosis at any point in time.  And for any particular shock. 
% % As a result it's possible to compute
% % these values for the ergodic distribution.
   \end{itemize}
 }




\frame{\frametitle{Ergodic Distribution Moments}
  \begin{itemize}
  \item Compute expected values for increasing values of t
  \item Note convergence characteristics
    \begin{itemize}
    \item $\parallel E_t(x_{t+k_1}) - E_t(x_{t+k_2}) \parallel , k_2>k_1$
    \item $ \parallel \spDrv{x_{t+k_2}}{x_{t-1}}  \parallel $
 %   \item $\left ( \spDrv{x_{t+k_2}}{x_{t-1}} \right ) $
    \end{itemize}
  \end{itemize}

}





\fwt{Better to Approximate h Subsequent to Perturbation Step}{
  \begin{itemize}
  \item Including auxiliary variables for each lead would be tedious
  \item Fortunately it is unnecessary
\item   The derivatives of g to nth order determine the derivatives of each
$h_k$ to nth order
\item Subsequent to the perturbation computation, we have all the components necessary for a {\bf Taylor Series Expansion} for $h_k$
  \end{itemize}


}
\section{Taylor Series Expansion for $x_{t+k}$}

\frame{\frametitle{Taylor Series Expansion for $x_{t+k}$}


For any time invariant recursively applied polynomial function:
\begin{itemize}
\item $x_{t+k}=H_k(x_{t-1},\epsilon_t, \ldots,\epsilon_{t+k})$
\item Given derivatives for $G$, the \faa\ formula provides derivatives for $H_k$
\item perturbation methods generate the required derivatives
\item projection methods derivatives available by inspection of terms
\end{itemize}

}

\frame{\frametitle{Functional Composition via the \faa\ Formula}
  
  \begin{itemize}
\item Given
 \begin{gather*}
\theComp
 \end{gather*}
\begin{gather*}
h_{\nu} = \sum_{m=1}^{{\vbar}\nu{\vbar}}\faDiPyr{m}{f} \gMultTerm{m}{g}
\intertext{where}
\gMultTerm{m}{g}=
   (\tnu!) \sum_{s=1}^m 
\sum_{p_s(\tnu,\tlam)}\prod_{j=1}^s \frac{(g_{\yell_j})^{\kay_j}}{(\kay_j!)((\yell_j!)^{{\vbar}\kay_j{\vbar}})}
\end{gather*}
\item 
Non linearity captured in the $\gMultTerm{m}{g}$.


  \end{itemize}

}


\section{Some Practical Considerations}
\frame{\frametitle{Some Practical Considerations}

\begin{itemize}
\item Umbral calculus techniques allow for more computationally efficient 
implementation\cite{nardo11}
\item No Normality assumption or integration required for computation of moments
\item no simulation
\item Note that the \faa\ Formula also applies for functions of future variables.
\begin{gather*}
  E_t(f(x_{t+k}))= E_t( f(h_k(g(x_{t-1},\epsilon_t),\epsilon_{t+1},\ldots,\epsilon_{t+k})))
\end{gather*}
\end{itemize}

}

\fwt{Formulae for Moments}{
  \begin{itemize}
\item Approximations for moments of future variables
\item $E_t(Var(x_{t+k})) \sim E_t(x_{t+k}^2)   - (E_t({t+k}))^2$
\item skewness
$E\left [ \left ( \frac{X-\mu}{\sigma} \right )^3 \right ] = \frac{E[X^3] - 3 \mu \sigma^2 - \mu^3}{\sigma^3}$
\item kurtosis
$E\left [ \left ( \frac{X-\mu}{\sigma} \right )^4 \right ] -3$
\item Can compute impulse response function characterizing the state variables 
distribution over time
\item Robustly characterize the evolution of qualitative and quantitative behavior
\item These values do not explode over time -- no need for pruning

  \end{itemize}


}


\frame{\frametitle{The Role of Symbolic Computation}
  \begin{itemize}
  \item Code consists of Mathematica (with a small amount of Java)
  \item No need for tensors or matrix derivative notation
  \item Integration of polynomial function of independently and identically distributed shocks amounts to pattern matching
  \item One could consider more complicated manipulation of shocks
    \begin{itemize}
    \item modes
\item  medians 
\item percentiles
    \end{itemize}
   \item Provides flexibility about transition from symbolic to numerical calculation
  \end{itemize}

}


%\frame{\frametitle{Umbral Calculus}}
% \frame{\frametitle{Multi-indices}

% The presentation and the computer code use multi-indices for describing the
% coefficients and in the code for determining where the coefficients will
% be stored.\cite{neidinger02}


% \begin{gather*}
% \mi{f}{\psi_1,\ldots,\psi_n}{x_1,\ldots,x_n}= \left [  \pDrv{x_1}{\psi_1} \cdots  \pDrv{x_n}{\psi_n} f \right ](x_1,\ldots,x_n)
% \end{gather*}

% GB Folland
% A multi-index is an n-tuple of nonnegative integers. Multi-indices are generally denoted
% by the
% the Greek letters 
% }
% \frame{

% If is a multi-index, we define
% \begin{gather*}
%   |\alpha|\ = \sum_{i=1}^n \alpha_i,  \alpha!= \Pi_{i=1}^n \alpha_i!\\
% x^\alpha = \Pi_{i=1}^n x_i^{\alpha_i} \intertext{ or }
%  \partial \alpha f = \partial_1^{\alpha_1} \partial_2^{\alpha_2} \cdots \partial_n^{\alpha_n}= \frac{\partial^{|\alpha|}}{\partial x_1^{\alpha_1} \partial x_2^{\alpha_2}\cdots \partial x_n^{\alpha_n}}
% \end{gather*}
% }
% \frame{

% The number $|\alpha|$ is called the order or degree of $\alpha$ Thus, the order of $\alpha$ is the same as the order of $x^\alpha$  as a monomial or 
% the order of $\partial \alpha $ as a partial derivative.
% kth-order partial derivative of f can be written simply as $\partial^\alpha f$ with $|\alpha = k|$.
% Consequently, multi-indices are very useful for wiritn both polynomials and partial derivatives.

% Neidinger's Algorithm2 provides a way to systematically step through
% the derivative pyramids.\cite{neidinger02}




% }


\frame{\frametitle{Ergodic Distribution Moments}
  \begin{itemize}
  \item Compute expected values for increasing values of t
  \item Note convergence characteristics
    \begin{itemize}
    \item $\parallel E_t(x_{t+k_1}) - E_t(x_{t+k_2}) \parallel , k_2>k_1$
    \item $ \parallel \spDrv{x_{t+k_2}}{x_{t-1}}  \parallel $
 %   \item $\left ( \spDrv{x_{t+k_2}}{x_{t-1}} \right ) $
    \end{itemize}
  \item ``Leapfrogging'' is useful
  \end{itemize}

}


\frame{\frametitle{Leapfrogging -- The Scalar Case}
{\small
\begin{gather*}
x_t=  g(x_{t-1},\epsilon_t)\\
E_t x_{t+1}=  E_t g(g(x_{t-1},\epsilon_t) \epsilon_{t+1})\\
E_t x_{t+2}=  E_t g(g(g(x_{t-1},\epsilon_t) \epsilon_{t+1}) \epsilon_{t+2})\\
E_t x_{t+2}= E_t h^2(g(x_{t-1},\epsilon_t),\epsilon_{t+1},\epsilon_{t+2})\intertext{ where}
h^2(x_t,\epsilon_{t+1},\epsilon_{t+2})
=g(g(x_t \epsilon_{t+1}) \epsilon_{t+2})\\
h^4(x_t,\epsilon_{t+1},\epsilon_{t+2},\epsilon_{t+3},\epsilon_{t+4})
=h^2(h^2(x_t \epsilon_{t+1} \epsilon_{t+2}) \epsilon_{t+3} \epsilon_{t+4})\\
\vdots\\
h^{2^k}(x_t,\epsilon_{t+1},\epsilon_{t+2},\ldots \epsilon_{t+2^k})
=h^{2^{k-1}}(h^{2^{k-1}}(x_t \epsilon_{t+1},\ldots  \epsilon_{t+2^{k-1}}) \epsilon_{t+2^{k-1}+1},\ldots  \epsilon_{t+2^k})\\
E_t x_{t+2^k}= E_t h^{2^k}(g(x_{t-1},\epsilon_t),\epsilon_{t+1},\ldots,\epsilon_{t+2^k})
\end{gather*}
}
}

\frame{\frametitle{Choosing a Linearization Point}
  \begin{itemize}
  \item Best point depends on loss function
  \item Uni-modal perhaps mode best
  \item Ergodic Mean Good Candidate
  \item Iterate on Linearization Point followed by Ergodic Mean calculation until convergence
  \end{itemize}
}

\section{Computing a Volterra Series}
\frame{\frametitle{Computing a Volterra Series}
{\small

  \begin{itemize}
  \item Recent paper\cite{meyer-gohde10,lan13} 
  \end{itemize}
    \begin{gather*}
0=E_t [ f(y_{t+1}^{fwdendo},y_t,y_{t-1}^{state},\epsilon_t)]\\
y_t(\sigma,\epsilon_{t},\epsilon_{t-1},\epsilon_{t-2},\ldots)\\
      y_t= \sum_{m=0}^M\frac{1}{m!} \vSum{1}\vSum{2}\cdots\vSum{m}\left [ \sum_{n=0}^{M-m} \frac{1}{n!} y_{\sigma^n i_1 i_2\cdots i_m}\sigma^n\right ] \vEps{1}\otimes \vEps{2} \otimes \cdots \otimes \vEps{m}
    \end{gather*}
}
}
% \frame{\frametitle{Mathematica Function}
%   \begin{itemize}
%   \item Compute a typical perturbation solution
%   \item Choose $m$ for the Volterra series
%   \item Compute a composition for the first time period
%   \item Compute 
%   \end{itemize}
% }


%\fwt{Mathematica Functions}{}





\frame{ \frametitle{intOut Mathematica Code}

{\tiny
\listinginput{1}{/msu/home/m1gsa00/RES2/conferences/sce13/code.mth}
}
}


\frame{\frametitle{Perturbation at Arbitrary Point}
{\small
\listinginput{1}{neoPertSoln.mth}
}
}

\bibliographystyle{authordate4}
\bibliography{files,anderson}
%\input{append.tex}	


\end{document}
