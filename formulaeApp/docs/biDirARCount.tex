
The AIM algorithm  requires at most
\begin{gather*}
  {L^3}\,\left( {\frac{4}{3}} + 26\,\left( \tau + \theta \right)  + 
    3\,{{\left( \tau + \theta \right) }^2} \right) 
\end{gather*}
 floating point operations
to produce the autoregressive representation, $\Gamma$.
Table \ref{tab:ARFlops} reports the floating point counts by type of operation.
The rank of $H^{\sharp,0}_\theta$ can be at most $L$. For typical problems,
the rank, after shifting obvious equations forward, is nearly L.  There
can be at most $L(\tau+\theta)$ shifted equations but  this number is typically
about equal to $L\theta$.

\begin{table*}[htbp]
\begin{minipage}{\textwidth}
  \begin{center}
    \leavevmode
    \begin{tabular}{|l|c|}
\hline
\multicolumn{1}{|c|}{Operation}&
\multicolumn{1}{|c|}{Floating Point Operations}\\
\hline
Initial QR Decomposition\footnote{$r$ matrix rank}&${\frac{4\,r\,\left( 3\,{L^2} - 3\,L\,r + {r^2} \right) }{3}}$\\
\hline
QR Rank One Update&$26\,{L^2}$\\
\hline
Multiplication to Annihilate Rows&${L^2}\,\left( \tau + \theta \right) $\\
\hline
Compute $\Gamma$&$2\,{L^2}\,\left( \tau + \theta \right) $\\
\hline
Max Total&${L^3}\,\left( {\frac{4}{3}} + 26\,\left( \tau + \theta \right)  + 
    3\,{{\left( \tau + \theta \right) }^2} \right) $\\
\hline
    \end{tabular}
    \caption{Floating Point Operations to Compute Autoregressive Representation}
    \label{tab:ARFlops}
  \end{center}
\end{minipage}
\end{table*}

