The example model 
produces $4(N-1)$  auxiliary initial conditions for the forward transition matrix, $4(N-1)-1\,\forall N>1$ auxiliary initial conditions for the reverse
transition matrix so that the zero invariant subspace has 
dimensions $8(N-1)-1$. Consequently, the minimal dimension
transition matrix has dimension $2N-1$.  We can carry out
eigenvalue calculations on a matrix of size $2N-1$ instead of $8(N-1)$. This can
significantly reduce computation time since
the floating point operations increase with the square of the matrix 
dimension.\footnote{For $N=1$, the larger matrix is  of dimension N and one can use the
auxiliary initial conditions to
demonstrate that all the eigenvalues are less one in magnitude without 
eigenvalue calculations.}


With $\alpha=2,\gamma={\frac{1}{10}},\theta=-{\frac{1}{5}}$,
 there are 
  $(N-1)$ roots larger than 1 in magnitude. 
Consequently, the algorithm determines
that the model has a unique solution converging to the steady state from arbitrary initial conditions so long as $Q_R^{-1}$ is full rank.
For example, with $N=2$
 there are 
  $1$ roots larger than 1 in magnitude. 
4 auxiliary initial conditions for the forward transition matrix, 3 auxiliary initial conditions for the reverse
transition matrix for a total of 7 dimension for the zero invariant subspace. Consequently, the minimal dimension
transition matrix has dimension $3$.  
